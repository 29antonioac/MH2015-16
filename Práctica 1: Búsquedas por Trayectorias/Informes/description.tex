El problema a resolver es el problema de \emph{Selección de características}.En el ámbito de la \emph{Ciencia de Datos}, la cantidad de datos a evaluar para obtener buenos resultados es excesivamente grande. Esto nos lleva a la siguiente cuestión: ¿Son todos ellos realmente importantes? ¿Podemos establecer dependencias para eliminar los que no nos aportan información relevante? La respuesta es que sí: en muchas ocasiones, no todos los datos son importantes, o no lo son demasiado. Por ello, se intentará filtrar las características relevantes de un conjunto de datos.

Para conseguir este propósito se deben usar técnicas probabilísticas, ya que es un problema \emph{NP-hard}. Una técnica exhaustiva sería totalmente inviable para cualquier caso de búsqueda medianamente grande.
