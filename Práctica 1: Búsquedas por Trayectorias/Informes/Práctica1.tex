\documentclass[a4paper, 11pt]{article}
\usepackage[utf8]{inputenc}
\usepackage{kvoptions-patch}
\usepackage[titulo={Práctica 1: Búsquedas con trayectorias simples}]{estilo}



\begin{document}
  \maketitle
  \tableofcontents
  \newpage

  \section{Descripción del problema}

    El problema a resolver es el problema de \emph{Selección de características}.En el ámbito de la \emph{Ciencia de Datos}, la cantidad de datos a evaluar para obtener buenos resultados es excesivamente grande. Esto nos lleva a la siguiente cuestión: ¿Son todos ellos realmente importantes? ¿Podemos establecer dependencias para eliminar los que no nos aportan información relevante? La respuesta es que sí: en muchas ocasiones, no todos los datos son importantes, o no lo son demasiado. Por ello, se intentará filtrar las características relevantes de un conjunto de datos.

La selección de características tiene varias ventajas: se reduce la complejidad del problema, disminuyendo el tiempo de ejecución. También se aumenta la capacidad de generalización puesto que tenemos menos variables que tener en cuenta, además de conseguir resultados más simples y fáciles de entender e interpretar.

Para conseguir este propósito se deben usar técnicas probabilísticas, ya que es un problema \emph{NP-hard}. Una técnica exhaustiva sería totalmente inviable para cualquier caso de búsqueda medianamente grande. Usaremos metaheurísticas para resolver este problema, aunque también podríamos intentar resolverlo utilizando estadísticos (correlación entre características, medidas de separabilidad o basadas en teoría de información o consistencia, etc).


  \section{Descripción de aplicación de los algoritmos al problema}
    Los elementos comunes de los algoritmos son:
    \begin{itemize}
      \item Representación de las soluciones: Se representan las soluciones como vectores 1-dimensionales binarios (los llamaremos \emph{bits} para poder hacer uso de términos como \emph{darle la vuelta a un bit}):

      $$ s = (x_1,x_2,\ldots,x_{n-1},x_n) ; \; x_i \in \{True,False\} \; \forall i \in \{1,2,\ldots,n\} $$
      \item Función objetivo: La función a maximizar es la tasa de clasificación de los datos de entrada:

      $$ tasa\_clas = 100 \cdot \frac{instancias\;bien\;clasificadas}{instancias\;totales} $$

      \item Generación de vecino: La función generadora de vecinos es bien simple. Se toma una solución y se le da la vuelta a uno de sus bits, el cual se escoge aleatoriamente.
      \begin{verbatim}
        Tomar una solución
        indice = generarAleatorio(0, numero_caracteristicas)
        caracteristicas[indice] = not caracteristicas[indice]
      \end{verbatim}
    \end{itemize}
  \section{Descripción de la estructura del método de búsqueda}

  \section{Descripción del algoritmo de comparación}

  \section{Desarrollo de la práctica}

  \section{Experimentos}

    Búsqueda Local\\
    \begin{table}[]
\centering
\caption{My caption}
\label{my-label}
\begin{tabular}{c|c|c|c|c|c|c|c|c|c|c|c|c|}
\cline{2-13}
                                    & \multicolumn{4}{c|}{Wdbc}             & \multicolumn{4}{c|}{Movement\_Libras} & \multicolumn{4}{c|}{Arrhythmia}       \\ \cline{2-13}
                                    & \% clas in & \% clas out & \% red & T & \% clas in & \% clas out & \% red & T & \% clas in & \% clas out & \% red & T \\ \hline
\multicolumn{1}{|c|}{ Partición 1-1 } &  97.88733    & 95.08772  & 43.33333  & 2.63815 & 71.66666 & 66.66667  & 41.11111  & 12.87521 & 65.625 & 68.04124 & 49.64029 & 41.70311  \\ \hline
\multicolumn{1}{|c|}{ Partición 1-2 } &  98.24561    & 96.47887  & 16.66667  & 3.09093 & 70.55556 & 80.0  & 52.22222  & 16.90017 & 67.01031 & 64.0625 & 46.76259 & 36.51126  \\ \hline
\multicolumn{1}{|c|}{ Media } &  98.06647    & 95.78330  & 30.00000  & 2.86454 & 71.11111 & 73.33334  & 46.66666  & 14.88769 & 66.31766 & 66.05187 & 48.20144 & 39.10719  \\ \hline
\end{tabular}
\end{table}

    SFS\\
    \resizebox{\textwidth}{!}{ \begin{tabular}{c|c|c|c|c|c|c|c|c|c|c|c|c|}
\cline{2-13}
                                    & \multicolumn{4}{c|}{Wdbc}             & \multicolumn{4}{c|}{Movement\_Libras} & \multicolumn{4}{c|}{Arrhythmia}       \\ \cline{2-13}
                                    & \% clas in & \% clas out & \% red & T & \% clas in & \% clas out & \% red & T & \% clas in & \% clas out & \% red & T \\ \hline
\multicolumn{1}{|c|}{ Partición 1-1 } &  97.53521    & 92.2807  & 83.33333  & 23.6847 & 75.0 & 61.11111  & 88.88889  & 86.85704 & 77.08333 & 69.07216 & 98.92086 & 113.45124  \\ \hline
\multicolumn{1}{|c|}{ Partición 1-2 } &  97.54386    & 93.66197  & 86.66667  & 21.27275 & 80.55556 & 73.88889  & 81.11111  & 133.14515 & 75.25773 & 67.70833 & 97.48201 & 221.32613  \\ \hline
\multicolumn{1}{|c|}{ Partición 2-1 } &  95.42254    & 91.22807  & 83.33333  & 24.21075 & 67.22222 & 57.77778  & 92.22222  & 60.01379 & 78.125 & 70.61856 & 97.84173 & 195.53327  \\ \hline
\multicolumn{1}{|c|}{ Partición 2-2 } &  95.78947    & 91.90141  & 93.33333  & 12.81063 & 76.11111 & 73.88889  & 92.22222  & 59.83782 & 79.38144 & 75.0 & 96.04317 & 325.50824  \\ \hline
\multicolumn{1}{|c|}{ Partición 3-1 } &  96.12676    & 92.98246  & 90.0  & 16.16634 & 80.0 & 68.88889  & 87.77778  & 87.70363 & 77.60417 & 70.61856 & 98.92086 & 106.32187  \\ \hline
\multicolumn{1}{|c|}{ Partición 3-2 } &  97.54386    & 96.47887  & 86.66667  & 20.40403 & 70.0 & 63.88889  & 91.11111  & 67.70326 & 80.41237 & 73.95833 & 96.40288 & 292.26432  \\ \hline
\multicolumn{1}{|c|}{ Partición 4-1 } &  98.23943    & 96.49123  & 86.66667  & 19.97793 & 73.33333 & 65.0  & 91.11111  & 68.92823 & 76.5625 & 71.64948 & 98.92086 & 105.00263  \\ \hline
\multicolumn{1}{|c|}{ Partición 4-2 } &  94.73684    & 94.3662  & 90.0  & 16.57442 & 69.44444 & 65.0  & 91.11111  & 73.90462 & 81.4433 & 76.5625 & 97.1223 & 243.91016  \\ \hline
\multicolumn{1}{|c|}{ Partición 5-1 } &  96.47887    & 92.63158  & 90.0  & 16.81309 & 73.33333 & 55.55556  & 94.44444  & 45.33103 & 72.91667 & 66.49485 & 99.28058 & 78.50098  \\ \hline
\multicolumn{1}{|c|}{ Partición 5-2 } &  98.94737    & 93.66197  & 76.66667  & 31.40311 & 62.77778 & 53.33333  & 93.33333  & 57.74021 & 77.83505 & 68.75 & 98.56115 & 134.39024  \\ \hline
\multicolumn{1}{|c|}{ Media } &  96.83642    & 93.56845  & 86.66667  & 20.33178 & 72.77778 & 63.83333  & 90.33333  & 74.11648 & 77.66216 & 71.04328 & 97.94964 & 181.62091  \\ \hline
\end{tabular} }


  \section{Referencias}

\end{document}
