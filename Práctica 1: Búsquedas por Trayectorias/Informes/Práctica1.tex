\documentclass[a4paper, 11pt]{article}
\usepackage[utf8]{inputenc}
\usepackage{kvoptions-patch}
\usepackage[titulo={Práctica 1: Búsquedas con trayectorias simples}]{estilo}



\begin{document}
  \maketitle
  \tableofcontents
  \newpage

  \section{Descripción del problema}

    El problema a resolver es el problema de \emph{Selección de características}.En el ámbito de la \emph{Ciencia de Datos}, la cantidad de datos a evaluar para obtener buenos resultados es excesivamente grande. Esto nos lleva a la siguiente cuestión: ¿Son todos ellos realmente importantes? ¿Podemos establecer dependencias para eliminar los que no nos aportan información relevante? La respuesta es que sí: en muchas ocasiones, no todos los datos son importantes, o no lo son demasiado. Por ello, se intentará filtrar las características relevantes de un conjunto de datos.

La selección de características tiene varias ventajas: se reduce la complejidad del problema, disminuyendo el tiempo de ejecución. También se aumenta la capacidad de generalización puesto que tenemos menos variables que tener en cuenta, además de conseguir resultados más simples y fáciles de entender e interpretar.

Para conseguir este propósito se deben usar técnicas probabilísticas, ya que es un problema \emph{NP-hard}. Una técnica exhaustiva sería totalmente inviable para cualquier caso de búsqueda medianamente grande. Usaremos metaheurísticas para resolver este problema, aunque también podríamos intentar resolverlo utilizando estadísticos (correlación entre características, medidas de separabilidad o basadas en teoría de información o consistencia, etc).


  \section{Descripción de aplicación de los algoritmos al problema}
    Los elementos comunes de los algoritmos son:
    \begin{itemize}
      \item Representación de las soluciones: Se representan las soluciones como vectores 1-dimensionales binarios (los llamaremos \emph{bits} para poder hacer uso de términos como \emph{darle la vuelta a un bit}):

      $$ s = (x_1,x_2,\ldots,x_{n-1},x_n) ; \; x_i \in \{True,False\} \; \forall i \in \{1,2,\ldots,n\} $$
      \item Función objetivo: La función a maximizar es la tasa de clasificación de los datos de entrada:

      $$ tasa\_clas = 100 \cdot \frac{instancias\;bien\;clasificadas}{instancias\;totales} $$

      \item Generación de vecino: La función generadora de vecinos es bien simple. Se toma una solución y se le da la vuelta a uno de sus bits, el cual se escoge aleatoriamente.
      \begin{verbatim}
        Tomar una solución
        indice = generarAleatorio(0, numero_caracteristicas)
        caracteristicas[indice] = not caracteristicas[indice]
      \end{verbatim}
    \end{itemize}
  \section{Descripción de la estructura del método de búsqueda}

  \section{Descripción del algoritmo de comparación}
    El algoritmo de comparación es un algoritmo greedy: el \emph{Sequential Forward Selection(SFS)}. La idea es muy simple: se parte del conjunto vacío de características (todos los bits a 0) y se recorren todas las características, evaluando la función de coste. La característica que más mejora ofrezca, se coje. Y se vuelve a empezar. Así hasta que ninguna de las características mejore el coste.

    \begin{verbatim}
      caracteristicas = (1,2,...,n)
      caracteristicas_seleccionadas = (0,0,...,0,0)
      fin = falso
      mejor_caracteristica = 0


      Mientras mejor_caracteristica != -1
        mejor_tasa = 0
        mejor_caracteristica = -1
        Para cada característica
          tasa = coste(característica)
          Si tasa > mejor_tasa
            mejor_tasa = tasa
            mejor_caracteristica = caracteristica
        Si mejor_caracteristica != -1
          caracteristicas_seleccionadas.añadir(mejor_caracteristica)
    \end{verbatim}
  \section{Desarrollo de la práctica}
    La práctica se ha desarrollado usando el lenguage de programación \emph{Python}, ya que su velocidad de desarrollo es bastante alta. Para intentar lidiar con la lentitud que puede suponer usar un lenguaje interpretado, utilizaremos las librerías \emph{NumPy, SciPy y Scikit-Learn}, que tienen módulos implementados en C (sobre todo \emph{NumPy}) y agilizan bastante los cálculos y el manejo de vectores grandes.

    Usaremos alguna funcionalidad directa de estas bibliotecas:
    \begin{itemize}
      \item \emph{NumPy}: Generación de números aleatorios y operaciones rápidas sobre vectores.
      \item \emph{SciPy}: Lectura de ficheros ARFF de WEKA.
      \item \emph{Scikit-Learn}: Particionamiento de los datos, tanto las particiones estratificadas de la validación cruzada 5x2 como las de \emph{Leave One Out} para calcular la función de coste. También se ha tomado un clasificador KNN, ya que está implementado usando estructuras de datos complejas como \emph{Ball Tree} y lo hace muy rápido.
    \end{itemize}

    Los requisitos para ejecutar mis prácticas son \emph{Python3} (importante que sea la 3), \emph{NumPy}, \emph{SciPy} y \emph{Scikit-Learn}. En mi plataforma (Archlinux) están disponibles desde su gestor de paquetes.

    Una vez instalados los paquetes, sólo hay que ejecutar la práctica diciéndole al programa los algoritmos que queremos ejecutar. La semilla aleatoria está fijada dentro del código como 12345678 para no inducir a errores. Veamos algunos ejemplos de llamadas a la práctica. Primero notamos que los algoritmos disponibles son:

    \begin{itemize}
      \item -SFS: Ejecuta el algoritmo greedy SFS.
      \item -LS: Ejecuta la Local Search.
      \item -SA: Ejecuta el Simulated Annealing.
      \item -TS: Ejecuta la Tabu Search.
      \item -TSext: Ejecuta la Tabu Search extendida.
    \end{itemize}

    \begin{verbatim}
      $ python practica1.py -TS
    \end{verbatim}
    Se ejecutará la Tabu Search. Pero no sólo se limita el programa a un algoritmo. Si le pasamos varios, los ejecutará en paralelo con el criterio de ejecutar tantos procesos como el mínimo entre los algoritmos pasados por argumento y el número de CPU's del sistema. Así, cada algoritmo podrá ir a una CPU distinta sin interferir en el rendimiento de ninguno de ellos.

    \begin{verbatim}
      $ python practica1.py -SFS -LS -SA
    \end{verbatim}
    Se ejecutarán en paralelo SFS, LS y SA paralelamente. Si se tuvieran dos núcleos, se ejecutarían los dos primeros y el tercero esperaría, para así no lastrar el rendimiento de ambos.

    Una vez ejecutado, irán saliendo por pantalla mensajes de este tipo, que proporcionan datos en tiempo real del estado de la ejecución:

    \begin{verbatim}
      INFO:__main__:W - TS - Time elapsed: 2265.526112794876.
      Score: 98.2394337654. Score out: 95.0877192982 Selected features: 15
    \end{verbatim}

    Este mensaje nos dice todo lo necesario: W es la base de datos (Wdbc), TS el algoritmo, el tiempo transcurrido para esta iteración (recordemos que hay 10), el score de entrenamiento, el score de validación y las características seleccionadas.
  \section{Experimentos}
    Como se ha comentado antes, la semilla está fija a 12345678 para no tener problemas de aleatoriedad. El número de evaluaciones máxima de todos los algoritmos es de 15000. Por lo demás, todos los demás parámetros propios de cada algoritmo están tal y como se explica en el guión ($\phi=\mu=0.3$, los valores de vecinos máximos, soluciones máximas aceptadas, etc). \\

    \centerline{SFS}
    \resizebox{\textwidth}{!}{ \begin{tabular}{c|c|c|c|c|c|c|c|c|c|c|c|c|}
\cline{2-13}
                                    & \multicolumn{4}{c|}{Wdbc}             & \multicolumn{4}{c|}{Movement\_Libras} & \multicolumn{4}{c|}{Arrhythmia}       \\ \cline{2-13}
                                    & \% clas in & \% clas out & \% red & T & \% clas in & \% clas out & \% red & T & \% clas in & \% clas out & \% red & T \\ \hline
\multicolumn{1}{|c|}{ Partición 1-1 } &  97.53521    & 92.2807  & 83.33333  & 23.6847 & 75.0 & 61.11111  & 88.88889  & 86.85704 & 77.08333 & 69.07216 & 98.92086 & 113.45124  \\ \hline
\multicolumn{1}{|c|}{ Partición 1-2 } &  97.54386    & 93.66197  & 86.66667  & 21.27275 & 80.55556 & 73.88889  & 81.11111  & 133.14515 & 75.25773 & 67.70833 & 97.48201 & 221.32613  \\ \hline
\multicolumn{1}{|c|}{ Partición 2-1 } &  95.42254    & 91.22807  & 83.33333  & 24.21075 & 67.22222 & 57.77778  & 92.22222  & 60.01379 & 78.125 & 70.61856 & 97.84173 & 195.53327  \\ \hline
\multicolumn{1}{|c|}{ Partición 2-2 } &  95.78947    & 91.90141  & 93.33333  & 12.81063 & 76.11111 & 73.88889  & 92.22222  & 59.83782 & 79.38144 & 75.0 & 96.04317 & 325.50824  \\ \hline
\multicolumn{1}{|c|}{ Partición 3-1 } &  96.12676    & 92.98246  & 90.0  & 16.16634 & 80.0 & 68.88889  & 87.77778  & 87.70363 & 77.60417 & 70.61856 & 98.92086 & 106.32187  \\ \hline
\multicolumn{1}{|c|}{ Partición 3-2 } &  97.54386    & 96.47887  & 86.66667  & 20.40403 & 70.0 & 63.88889  & 91.11111  & 67.70326 & 80.41237 & 73.95833 & 96.40288 & 292.26432  \\ \hline
\multicolumn{1}{|c|}{ Partición 4-1 } &  98.23943    & 96.49123  & 86.66667  & 19.97793 & 73.33333 & 65.0  & 91.11111  & 68.92823 & 76.5625 & 71.64948 & 98.92086 & 105.00263  \\ \hline
\multicolumn{1}{|c|}{ Partición 4-2 } &  94.73684    & 94.3662  & 90.0  & 16.57442 & 69.44444 & 65.0  & 91.11111  & 73.90462 & 81.4433 & 76.5625 & 97.1223 & 243.91016  \\ \hline
\multicolumn{1}{|c|}{ Partición 5-1 } &  96.47887    & 92.63158  & 90.0  & 16.81309 & 73.33333 & 55.55556  & 94.44444  & 45.33103 & 72.91667 & 66.49485 & 99.28058 & 78.50098  \\ \hline
\multicolumn{1}{|c|}{ Partición 5-2 } &  98.94737    & 93.66197  & 76.66667  & 31.40311 & 62.77778 & 53.33333  & 93.33333  & 57.74021 & 77.83505 & 68.75 & 98.56115 & 134.39024  \\ \hline
\multicolumn{1}{|c|}{ Media } &  96.83642    & 93.56845  & 86.66667  & 20.33178 & 72.77778 & 63.83333  & 90.33333  & 74.11648 & 77.66216 & 71.04328 & 97.94964 & 181.62091  \\ \hline
\end{tabular} }

    \\ \centerline{Búsqueda Local}
    \begin{table}[]
\centering
\caption{My caption}
\label{my-label}
\begin{tabular}{c|c|c|c|c|c|c|c|c|c|c|c|c|}
\cline{2-13}
                                    & \multicolumn{4}{c|}{Wdbc}             & \multicolumn{4}{c|}{Movement\_Libras} & \multicolumn{4}{c|}{Arrhythmia}       \\ \cline{2-13}
                                    & \% clas in & \% clas out & \% red & T & \% clas in & \% clas out & \% red & T & \% clas in & \% clas out & \% red & T \\ \hline
\multicolumn{1}{|c|}{ Partición 1-1 } &  97.88733    & 95.08772  & 43.33333  & 2.63815 & 71.66666 & 66.66667  & 41.11111  & 12.87521 & 65.625 & 68.04124 & 49.64029 & 41.70311  \\ \hline
\multicolumn{1}{|c|}{ Partición 1-2 } &  98.24561    & 96.47887  & 16.66667  & 3.09093 & 70.55556 & 80.0  & 52.22222  & 16.90017 & 67.01031 & 64.0625 & 46.76259 & 36.51126  \\ \hline
\multicolumn{1}{|c|}{ Media } &  98.06647    & 95.78330  & 30.00000  & 2.86454 & 71.11111 & 73.33334  & 46.66666  & 14.88769 & 66.31766 & 66.05187 & 48.20144 & 39.10719  \\ \hline
\end{tabular}
\end{table}
\\
    \\ \centerline{SA}
    \resizebox{\textwidth}{!}{ \begin{tabular}{c|c|c|c|c|c|c|c|c|c|c|c|c|}
\cline{2-13}
                                    & \multicolumn{4}{c|}{Wdbc}             & \multicolumn{4}{c|}{Movement\_Libras} & \multicolumn{4}{c|}{Arrhythmia}       \\ \cline{2-13}
                                    & \% clas in & \% clas out & \% red & T & \% clas in & \% clas out & \% red & T & \% clas in & \% clas out & \% red & T \\ \hline
\multicolumn{1}{|c|}{ Partición 1-1 } &  97.53521    & 96.49123  & 40.0  & 100.29172 & 70.0 & 67.22222  & 46.66667  & 213.54383 & 71.35417 & 68.04124 & 53.59712 & 855.73377  \\ \hline
\multicolumn{1}{|c|}{ Partición 1-2 } &  97.89473    & 96.83099  & 63.33333  & 95.93822 & 70.55556 & 77.77778  & 51.11111  & 212.27381 & 67.52577 & 62.5 & 43.16547 & 894.88636  \\ \hline
\multicolumn{1}{|c|}{ Partición 2-1 } &  97.53521    & 94.38596  & 50.0  & 97.93278 & 73.88889 & 63.33333  & 48.88889  & 212.83528 & 69.27083 & 65.97938 & 53.59712 & 824.82599  \\ \hline
\multicolumn{1}{|c|}{ Partición 2-2 } &  96.49123    & 95.07042  & 36.66667  & 98.17512 & 72.22222 & 73.88889  & 47.77778  & 213.95732 & 69.58763 & 64.58333 & 52.51799 & 850.47581  \\ \hline
\multicolumn{1}{|c|}{ Partición 3-1 } &  97.1831    & 96.14035  & 26.66667  & 98.46561 & 75.0 & 71.11111  & 53.33333  & 211.40456 & 70.3125 & 66.49485 & 45.68345 & 1016.30123  \\ \hline
\multicolumn{1}{|c|}{ Partición 3-2 } &  97.19298    & 95.42254  & 40.0  & 98.32049 & 69.44444 & 71.11111  & 46.66667  & 210.78197 & 73.19587 & 64.0625 & 50.0 & 1001.46568  \\ \hline
\multicolumn{1}{|c|}{ Partición 4-1 } &  98.94366    & 93.33333  & 46.66667  & 97.33434 & 73.33333 & 73.88889  & 46.66667  & 212.77572 & 70.3125 & 63.91753 & 49.64029 & 948.78478  \\ \hline
\multicolumn{1}{|c|}{ Partición 4-2 } &  96.14035    & 96.83099  & 30.0  & 113.81931 & 65.55555 & 74.44444  & 55.55556  & 211.13621 & 68.5567 & 60.9375 & 46.76259 & 979.51487  \\ \hline
\multicolumn{1}{|c|}{ Partición 5-1 } &  96.83099    & 94.38596  & 46.66667  & 110.18032 & 71.66666 & 65.0  & 38.88889  & 216.25208 & 68.75 & 64.43299 & 52.8777 & 851.86418  \\ \hline
\multicolumn{1}{|c|}{ Partición 5-2 } &  98.24561    & 96.12676  & 50.0  & 109.40455 & 72.77778 & 74.44444  & 48.88889  & 191.48629 & 70.10309 & 69.27083 & 56.11511 & 919.57164  \\ \hline
\multicolumn{1}{|c|}{ Media } &  97.39931    & 95.50185  & 43.00000  & 101.98625 & 71.44444 & 71.22222  & 48.44445  & 210.64471 & 69.89691 & 65.02201 & 50.39568 & 914.34243  \\ \hline
\end{tabular} }


    El caso de la búsqueda tabú extendida no lo tengo completo: Por alguna razón, en la 5º ejecución del algoritmo en la base de datos \emph{Arrhythmia} la ejecución me ofrecía un \emph{Memory Error}, después de más de 12 horas ejecutando.

  \section{Referencias}

\end{document}
