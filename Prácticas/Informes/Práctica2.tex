\documentclass[a4paper, 11pt]{article}
\usepackage[utf8]{inputenc}
\usepackage{kvoptions-patch}
\usepackage[titulo={Práctica 2: Búsquedas con trayectorias múltiples}]{estilo}



\begin{document}
  \maketitle
  \tableofcontents
  \newpage

  \section{Descripción del problema}

    El problema a resolver es el problema de \emph{Selección de características}.En el ámbito de la \emph{Ciencia de Datos}, la cantidad de datos a evaluar para obtener buenos resultados es excesivamente grande. Esto nos lleva a la siguiente cuestión: ¿Son todos ellos realmente importantes? ¿Podemos establecer dependencias para eliminar los que no nos aportan información relevante? La respuesta es que sí: en muchas ocasiones, no todos los datos son importantes, o no lo son demasiado. Por ello, se intentará filtrar las características relevantes de un conjunto de datos.

La selección de características tiene varias ventajas: se reduce la complejidad del problema, disminuyendo el tiempo de ejecución. También se aumenta la capacidad de generalización puesto que tenemos menos variables que tener en cuenta, además de conseguir resultados más simples y fáciles de entender e interpretar.

Para conseguir este propósito se deben usar técnicas probabilísticas, ya que es un problema \emph{NP-hard}. Una técnica exhaustiva sería totalmente inviable para cualquier caso de búsqueda medianamente grande. Usaremos metaheurísticas para resolver este problema, aunque también podríamos intentar resolverlo utilizando estadísticos (correlación entre características, medidas de separabilidad o basadas en teoría de información o consistencia, etc).


  \section{Descripción de aplicación de los algoritmos al problema}
    Los elementos comunes de los algoritmos son:
\begin{itemize}
  \item Representación de las soluciones: Se representan las soluciones como vectores 1-dimensionales binarios (los llamaremos \emph{bits} para poder hacer uso de términos como \emph{darle la vuelta a un bit}):

  $$ s = (x_1,x_2,\ldots,x_{n-1},x_n) ; \; x_i \in \{True,False\} \; \forall i \in \{1,2,\ldots,n\} $$
  \item Función objetivo: La función a maximizar es la tasa de clasificación de los datos de entrada:

  $$ tasa\_clas = 100 \cdot \frac{instancias\;bien\;clasificadas}{instancias\;totales} $$

  \item Generación de vecino: La función generadora de vecinos es bien simple. Se toma una solución y se le da la vuelta a uno de sus bits, el cual se escoge aleatoriamente.
  \begin{verbatim}
    Tomar un vector de características "caracteristica"
    indice = generarAleatorio(0, numero_caracteristicas)
    caracteristicas[indice] = not caracteristicas[indice]
  \end{verbatim}
\end{itemize}

  \section{Descripción de la estructura del método de búsqueda}
    Veamos los esquemas de cada algoritmo en pseudocódigo:
    \begin{itemize}
      \item Búsqueda Multiarranque Básica:
        \begin{verbatim}
          mejor_solucion = [False,False,...,False]
          mejor_tasa = 0
          numero_busquedas = 25

          Desde 1 hasta numero_busquedas
              caracteristicas_seleccionadas, tasa = LS(datos_entrenamiento,
                  etiquetas_entrenamiento, clasificador)

              Si tasa > mejor_tasa:
                  mejor_tasa = tasa
                  mejor_solucion = caracteristicas_seleccionadas

          return mejor_solucion, mejor_tasa
        \end{verbatim}
      \item GRASP:
      \begin{verbatim}
        mejor_solucion = [False,False,...,False]
        mejor_tasa = 0
        numero_busquedas = 25

        Desde 1 hasta numero_busquedas
            caracteristicas_seleccionadas, tasa = SFSrandom(datos_entrenamiento,
                etiquetas_entrenamiento, clasificador)
            caracteristicas_seleccionadas, tasa = LS(datos_entrenamiento,
                etiquetas_entrenamiento, clasificador, caracteristicas_seleccionadas)

            Si tasa > mejor_tasa:
                mejor_tasa = tasa
                mejor_solucion = caracteristicas_seleccionadas

        return mejor_solucion, mejor_tasa
      \end{verbatim}
      \item Iterated Local Search:
      \begin{verbatim}
        solucion_inicial = solucionAleatoria()
        mejor_solucion = solucion_inicial
        num_searchs = 25
        mejor_tasa = 0

        caracteristicas_seleccionadas, _ = LS(datos_entrenamiento,
            etiquetas_entrenamiento, clasificador, solucion_inicial)

        Desde 1 hasta numero_busquedas-1
            nuevas_caracteristicas, nueva_tasa = LS(datos_entrenamiento,
                etiquetas_entrenamiento, clasificador,
                 mutacion(caracteristicas_seleccionadas))

            Si nueva_tasa > mejor_tasa:
                mejor_tasa = nueva_tasa
                mejor_solucion = nuevas_caracteristicas

        return mejor_solucion, mejor_tasa
      \end{verbatim}
    \end{itemize}

    Para el GRASP se ha tenido que implementar una versión aleatorizada del SFS, la cual mostramos aquí:

    \begin{verbatim}
      caracteristicas_seleccionadas = [False,False,...,False]
      mejor_tasa_temporal = 0
      peor_tasa_temporal = 0
      mejor_caracteristica = 0
      mejor_tasa = 0
      alpha = 0.3

      Mientras mejor_caracteristica no sea None
          mejor_caracteristica = None

          caracteristicas_disponibles = Índices de caracteristicas de
            caracteristicas_seleccionadas que están a False
          tasas = [0,0,...,0]
          caracteristicas_restringidas = ListaVacía

          ### Enumeración devuelve las características con su índice en el vector
          Para idx,data_idx en enumeración(caracteristicas_disponibles)

              caracteristicas_seleccionadas[data_idx] = True
              tasas[idx] = clasificador.tasarSolucion(caracteristicas_seleccionadas)
              caracteristicas_seleccionadas[data_idx] = False

              Si tasas[idx] > mejor_tasa_temporal
                  mejor_tasa_temporal = tasas[idx]
              Si no tasas[idx] < peor_tasa_temporal
                  peor_tasa_temporal = tasas[idx]

          Para idx,data_idx en enumeración(caracteristicas_disponibles)
              Si tasas[idx] > umbral
                  caracteristicas_restringidas.añadir(data_idx)

          caracteristica_aleatoria = aleatorio de caracteristicas_restringidas

          caracteristicas_seleccionadas[caracteristica_aleatoria] = True
          tasa = clasificador.tasarSolucion(caracteristicas_seleccionadas)

          Si tasa > mejor_tasa
              mejor_tasa = tasa
              mejor_caracteristica = caracteristica_aleatoria
          En otro caso
              caracteristicas_seleccionadas[caracteristica_aleatoria] = False

      return caracteristicas_seleccionadas, mejor_tasa
    \end{verbatim}

    Para el ILS el operador de mutación es darle la vuelta al 10\% de los bits de la máscara. Se ha implementado así:

    \begin{verbatim}
      cambios = EnteroPorArriba(0.1 * longitud(caracteristicas))
      mascara = repetir(True, cambios)
      intactos = repetir(False, longitud(caracteristicas) - cambios)
      mascara_completa = concatenar((mascara,intactos))
      Shuffle(mascara_completa)
      caracteristicas_mutadas = np.logical_xor(caracteristicas,mascara_completa)
      return caracteristicas_mutadas
    \end{verbatim}
  \section{Descripción del algoritmo de comparación}
    El algoritmo de comparación es un algoritmo greedy: el \emph{Sequential Forward Selection(SFS)}. La idea es muy simple: se parte del conjunto vacío de características (todos los bits a 0) y se recorren todas las características, evaluando la función de coste. La característica que más mejora ofrezca, se coje. Y se vuelve a empezar. Así hasta que ninguna de las características mejore el coste.

\begin{verbatim}
  caracteristicas = (1,2,...,n)
  caracteristicas_seleccionadas = (0,0,...,0,0)
  fin = falso
  mejor_caracteristica = 0


  Mientras mejor_caracteristica != -1
    mejor_tasa = 0
    mejor_caracteristica = -1
    Para cada característica
      tasa = coste(característica)
      Si tasa > mejor_tasa
        mejor_tasa = tasa
        mejor_caracteristica = caracteristica
    Si mejor_caracteristica != -1
      caracteristicas_seleccionadas.añadir(mejor_caracteristica)
\end{verbatim}

  \section{Desarrollo de la práctica}
    En primer lugar, comentar que las bases de datos han sido modificadas en su estructura (que no en sus datos) para que sean homogéneas. Así, se han puesto todas las clases como numéricas (en Wdbc no lo estaban) y se han colocado en la última columna.

    La práctica se ha desarrollado usando el lenguage de programación \emph{Python}, ya que su velocidad de desarrollo es bastante alta. Para intentar lidiar con la lentitud que puede suponer usar un lenguaje interpretado, utilizaremos las librerías \emph{NumPy, SciPy y Scikit-Learn}, que tienen módulos implementados en C (sobre todo \emph{NumPy}) y agilizan bastante los cálculos y el manejo de vectores grandes.

    Usaremos alguna funcionalidad directa de estas bibliotecas:
    \begin{itemize}
      \item \emph{NumPy}: Generación de números aleatorios y operaciones rápidas sobre vectores.
      \item \emph{SciPy}: Lectura de ficheros ARFF de WEKA.
      \item \emph{Scikit-Learn}: Particionamiento de los datos, tanto las particiones estratificadas de la validación cruzada 5x2 como las de \emph{Leave One Out} para calcular la función de coste. También se ha tomado un clasificador KNN, ya que está implementado usando estructuras de datos complejas como \emph{Ball Tree} y lo hace muy rápido.
    \end{itemize}

    Esta elección se ha hecho para poder preocuparme sólo y exclusivamente de la implementación de las metaheurísticas.

    Los requisitos para ejecutar mis prácticas son \emph{Python3} (importante que sea la 3), \emph{NumPy}, \emph{SciPy} y \emph{Scikit-Learn}. En mi plataforma (Archlinux) están disponibles desde su gestor de paquetes.

    Una vez instalados los paquetes, sólo hay que ejecutar la práctica diciéndole al programa los algoritmos que queremos ejecutar. La semilla aleatoria está fijada dentro del código como 12345678 para no inducir a errores. Veamos algunos ejemplos de llamadas a la práctica. Primero notamos que los algoritmos disponibles son:

    \begin{itemize}
      \item -SFS: Ejecuta el algoritmo greedy SFS.
      \item -LS: Ejecuta la Local Search.
      \item -SA: Ejecuta el Simulated Annealing.
      \item -TS: Ejecuta la Tabu Search.
      \item -TSext: Ejecuta la Tabu Search extendida.
      \item -BMB: Ejecuta la Búsqueda Multiarranque Básica.
      \item -GRASP: Ejecuta el GRASP.
      \item -ILS: Ejecuta la Iterated Local Search
    \end{itemize}

    \begin{verbatim}
      $ python featureSelection.py -TS
    \end{verbatim}
    Se ejecutará la Tabu Search. Pero no sólo se limita el programa a un algoritmo. Si le pasamos varios, los ejecutará en serie uno detrás de otro. Esto ha cambiado desde la práctica anterior por la entrada de CUDA, que hay que iniciarlo debidamente y no es tan sencillo de ejecutar cosas en paralelo.

    \begin{verbatim}
      $ python featureSelection.py -BMB -GRASP -ILS
    \end{verbatim}
    Se ejecutarán en paralelo BMB, GRASP e ILS en serie.

    Una vez ejecutado, irán saliendo por pantalla mensajes de este tipo, que proporcionan datos en tiempo real del estado de la ejecución:

    \begin{verbatim}
      INFO:__main__:W - TS - Time elapsed: 2265.526112794876.
      Score: 98.2394337654. Score out: 95.0877192982 Selected features: 15
    \end{verbatim}

    Este mensaje nos dice todo lo necesario: W es la base de datos (Wdbc), TS el algoritmo, el tiempo transcurrido para esta iteración (recordemos que hay 10), el score de entrenamiento, el score de validación y las características seleccionadas.
  \section{Experimentos}
    Como se ha comentado antes, la semilla está fija a 12345678 para no tener problemas de aleatoriedad. El número de evaluaciones máxima de todos los algoritmos es de 15000. Por lo demás, todos los demás parámetros propios de cada algoritmo están tal y como se explica en el guión ($\mu=0.3$, los valores de vecinos máximos, soluciones máximas aceptadas, etc). \\

    \newpage

    \centerline{KNN}
    \resizebox{\textwidth}{!}{ \begin{tabular}{c|c|c|c|c|c|c|c|c|c|c|c|c|}
\cline{2-13}
                                    & \multicolumn{4}{c|}{Wdbc}             & \multicolumn{4}{c|}{Movement\_Libras} & \multicolumn{4}{c|}{Arrhythmia}       \\ \cline{2-13}
                                    & \% clas in & \% clas out & \% red & T & \% clas in & \% clas out & \% red & T & \% clas in & \% clas out & \% red & T \\ \hline
\multicolumn{1}{|c|}{ Partición 1-1 } &  96.12676    & 96.84211  & 0.0  & 0.0 & 67.77778 & 65.0  & 0.0  & 0.0 & 64.58333 & 66.49485 & 0.0 & 0.0  \\ \hline
\multicolumn{1}{|c|}{ Partición 1-2 } &  96.49123    & 96.83099  & 0.0  & 0.0 & 66.66667 & 80.55556  & 0.0  & 0.0 & 63.40206 & 62.5 & 0.0 & 0.0  \\ \hline
\multicolumn{1}{|c|}{ Partición 2-1 } &  96.12676    & 96.49123  & 0.0  & 0.0 & 66.11111 & 74.44444  & 0.0  & 0.0 & 64.0625 & 64.43299 & 0.0 & 0.0  \\ \hline
\multicolumn{1}{|c|}{ Partición 2-2 } &  95.78947    & 96.12676  & 0.0  & 0.0 & 70.55556 & 68.88889  & 0.0  & 0.0 & 64.94845 & 64.58333 & 0.0 & 0.0  \\ \hline
\multicolumn{1}{|c|}{ Partición 3-1 } &  96.47887    & 95.4386  & 0.0  & 0.0 & 76.11111 & 71.66667  & 0.0  & 0.0 & 64.0625 & 63.91753 & 0.0 & 0.0  \\ \hline
\multicolumn{1}{|c|}{ Partición 3-2 } &  96.8421    & 96.83099  & 0.0  & 0.0 & 66.66667 & 65.55556  & 0.0  & 0.0 & 63.40206 & 64.58333 & 0.0 & 0.0  \\ \hline
\multicolumn{1}{|c|}{ Partición 4-1 } &  97.53521    & 96.14035  & 0.0  & 0.0 & 65.0 & 66.11111  & 0.0  & 0.0 & 67.1875 & 64.43299 & 0.0 & 0.0  \\ \hline
\multicolumn{1}{|c|}{ Partición 4-2 } &  93.33333    & 97.88732  & 0.0  & 0.0 & 70.0 & 72.77778  & 0.0  & 0.0 & 62.8866 & 64.58333 & 0.0 & 0.0  \\ \hline
\multicolumn{1}{|c|}{ Partición 5-1 } &  96.47887    & 96.84211  & 0.0  & 0.0 & 71.66666 & 71.11111  & 0.0  & 0.0 & 65.10417 & 65.97938 & 0.0 & 0.0  \\ \hline
\multicolumn{1}{|c|}{ Partición 5-2 } &  97.89473    & 95.42254  & 0.0  & 0.0 & 65.55555 & 70.0  & 0.0  & 0.0 & 63.40206 & 63.54167 & 0.0 & 0.0  \\ \hline
\multicolumn{1}{|c|}{ Media } &  96.30973    & 96.48530  & 0.00000  & 0.0 & 68.61111 & 70.61111  & 0.00000  & 0.0 & 64.30412 & 64.50494 & 0.00000 & 0.0  \\ \hline
\end{tabular} }
\\
    \centerline{SFS}
    \resizebox{\textwidth}{!}{ \begin{tabular}{c|c|c|c|c|c|c|c|c|c|c|c|c|}
\cline{2-13}
                                    & \multicolumn{4}{c|}{Wdbc}             & \multicolumn{4}{c|}{Movement\_Libras} & \multicolumn{4}{c|}{Arrhythmia}       \\ \cline{2-13}
                                    & \% clas in & \% clas out & \% red & T & \% clas in & \% clas out & \% red & T & \% clas in & \% clas out & \% red & T \\ \hline
\multicolumn{1}{|c|}{ Partición 1-1 } &  97.53521    & 92.2807  & 83.33333  & 23.6847 & 75.0 & 61.11111  & 88.88889  & 86.85704 & 77.08333 & 69.07216 & 98.92086 & 113.45124  \\ \hline
\multicolumn{1}{|c|}{ Partición 1-2 } &  97.54386    & 93.66197  & 86.66667  & 21.27275 & 80.55556 & 73.88889  & 81.11111  & 133.14515 & 75.25773 & 67.70833 & 97.48201 & 221.32613  \\ \hline
\multicolumn{1}{|c|}{ Partición 2-1 } &  95.42254    & 91.22807  & 83.33333  & 24.21075 & 67.22222 & 57.77778  & 92.22222  & 60.01379 & 78.125 & 70.61856 & 97.84173 & 195.53327  \\ \hline
\multicolumn{1}{|c|}{ Partición 2-2 } &  95.78947    & 91.90141  & 93.33333  & 12.81063 & 76.11111 & 73.88889  & 92.22222  & 59.83782 & 79.38144 & 75.0 & 96.04317 & 325.50824  \\ \hline
\multicolumn{1}{|c|}{ Partición 3-1 } &  96.12676    & 92.98246  & 90.0  & 16.16634 & 80.0 & 68.88889  & 87.77778  & 87.70363 & 77.60417 & 70.61856 & 98.92086 & 106.32187  \\ \hline
\multicolumn{1}{|c|}{ Partición 3-2 } &  97.54386    & 96.47887  & 86.66667  & 20.40403 & 70.0 & 63.88889  & 91.11111  & 67.70326 & 80.41237 & 73.95833 & 96.40288 & 292.26432  \\ \hline
\multicolumn{1}{|c|}{ Partición 4-1 } &  98.23943    & 96.49123  & 86.66667  & 19.97793 & 73.33333 & 65.0  & 91.11111  & 68.92823 & 76.5625 & 71.64948 & 98.92086 & 105.00263  \\ \hline
\multicolumn{1}{|c|}{ Partición 4-2 } &  94.73684    & 94.3662  & 90.0  & 16.57442 & 69.44444 & 65.0  & 91.11111  & 73.90462 & 81.4433 & 76.5625 & 97.1223 & 243.91016  \\ \hline
\multicolumn{1}{|c|}{ Partición 5-1 } &  96.47887    & 92.63158  & 90.0  & 16.81309 & 73.33333 & 55.55556  & 94.44444  & 45.33103 & 72.91667 & 66.49485 & 99.28058 & 78.50098  \\ \hline
\multicolumn{1}{|c|}{ Partición 5-2 } &  98.94737    & 93.66197  & 76.66667  & 31.40311 & 62.77778 & 53.33333  & 93.33333  & 57.74021 & 77.83505 & 68.75 & 98.56115 & 134.39024  \\ \hline
\multicolumn{1}{|c|}{ Media } &  96.83642    & 93.56845  & 86.66667  & 20.33178 & 72.77778 & 63.83333  & 90.33333  & 74.11648 & 77.66216 & 71.04328 & 97.94964 & 181.62091  \\ \hline
\end{tabular} }

    \\ \centerline{BMB}
    \resizebox{\textwidth}{!}{ \begin{tabular}{c|c|c|c|c|c|c|c|c|c|c|c|c|}
\cline{2-13}
                                    & \multicolumn{4}{c|}{Wdbc}             & \multicolumn{4}{c|}{Movement\_Libras} & \multicolumn{4}{c|}{Arrhythmia}       \\ \cline{2-13}
                                    & \% clas in & \% clas out & \% red & T & \% clas in & \% clas out & \% red & T & \% clas in & \% clas out & \% red & T \\ \hline
\multicolumn{1}{|c|}{ Partición 1-1 } &  98.23944    & 97.19298  & 26.66667  & 1.09317 & 73.33333 & 66.11111  & 47.77778  & 4.67327 & 70.83333 & 67.01031 & 44.96403 & 76.77209  \\ \hline
\multicolumn{1}{|c|}{ Partición 1-2 } &  97.89474    & 94.71831  & 46.66667  & 1.1125 & 72.77778 & 76.11111  & 58.88889  & 4.2988 & 69.58763 & 62.5 & 47.84173 & 58.7279  \\ \hline
\multicolumn{1}{|c|}{ Partición 2-1 } &  97.53521    & 96.49123  & 36.66667  & 1.12211 & 80.55556 & 63.33333  & 57.77778  & 4.23303 & 70.3125 & 63.91753 & 49.64029 & 71.94003  \\ \hline
\multicolumn{1}{|c|}{ Partición 2-2 } &  97.89474    & 95.42254  & 43.33333  & 1.15543 & 74.44444 & 73.88889  & 47.77778  & 4.35542 & 70.61856 & 65.10417 & 46.76259 & 64.8806  \\ \hline
\multicolumn{1}{|c|}{ Partición 3-1 } &  98.23944    & 97.19298  & 20.0  & 1.05712 & 83.33333 & 64.44444  & 45.55556  & 4.49277 & 73.4375 & 64.43299 & 51.07914 & 73.12431  \\ \hline
\multicolumn{1}{|c|}{ Partición 3-2 } &  97.19298    & 95.77465  & 36.66667  & 1.11297 & 75.55556 & 74.44444  & 46.66667  & 4.18188 & 69.58763 & 66.14583 & 46.40288 & 51.61347  \\ \hline
\multicolumn{1}{|c|}{ Partición 4-1 } &  97.88732    & 94.38596  & 30.0  & 1.00256 & 77.22222 & 70.0  & 45.55556  & 4.36173 & 72.39583 & 65.46392 & 50.0 & 64.6652  \\ \hline
\multicolumn{1}{|c|}{ Partición 4-2 } &  98.24561    & 94.71831  & 46.66667  & 1.12321 & 76.11111 & 73.88889  & 50.0  & 4.23004 & 72.16495 & 63.02083 & 43.16547 & 58.72289  \\ \hline
\multicolumn{1}{|c|}{ Partición 5-1 } &  98.23944    & 94.73684  & 43.33333  & 1.06779 & 75.55556 & 75.55556  & 43.33333  & 4.44234 & 69.79167 & 68.04124 & 49.28058 & 72.01491  \\ \hline
\multicolumn{1}{|c|}{ Partición 5-2 } &  97.89474    & 96.12676  & 30.0  & 1.01396 & 77.77778 & 73.88889  & 36.66667  & 4.53782 & 71.64948 & 69.27083 & 48.56115 & 63.70405  \\ \hline
\multicolumn{1}{|c|}{ Media } &  97.92637    & 95.67606  & 36.00000  & 1.08608 & 76.66667 & 71.16667  & 48.00000  & 4.38071 & 71.03791 & 65.49077 & 47.76979 & 65.61655  \\ \hline
\end{tabular} }
\\
    \\ \centerline{GRASP}
    \resizebox{\textwidth}{!}{ \begin{tabular}{c|c|c|c|c|c|c|c|c|c|c|c|c|}
\cline{2-13}
                                    & \multicolumn{4}{c|}{Wdbc}             & \multicolumn{4}{c|}{Movement\_Libras} & \multicolumn{4}{c|}{Arrhythmia}       \\ \cline{2-13}
                                    & \% clas in & \% clas out & \% red & T & \% clas in & \% clas out & \% red & T & \% clas in & \% clas out & \% red & T \\ \hline
\multicolumn{1}{|c|}{ Partición 1-1 } &  98.94366    & 96.84211  & 73.33333  & 3.96508 & 79.44444 & 70.0  & 77.77778  & 16.53877 & 83.85417 & 65.97938 & 90.28777 & 34.61972  \\ \hline
\multicolumn{1}{|c|}{ Partición 1-2 } &  98.24561    & 96.83099  & 63.33333  & 4.25855 & 80.0 & 85.55556  & 83.33333  & 18.53391 & 79.38144 & 61.45833 & 89.56835 & 35.33348  \\ \hline
\multicolumn{1}{|c|}{ Partición 2-1 } &  97.88732    & 96.14035  & 63.33333  & 3.48653 & 80.0 & 66.66667  & 83.33333  & 18.17763 & 84.89583 & 63.91753 & 92.80576 & 38.46195  \\ \hline
\multicolumn{1}{|c|}{ Partición 2-2 } &  97.89474    & 96.47887  & 63.33333  & 4.35416 & 78.33333 & 71.11111  & 84.44444  & 15.93595 & 80.41237 & 63.02083 & 90.28777 & 32.37455  \\ \hline
\multicolumn{1}{|c|}{ Partición 3-1 } &  97.88732    & 95.78947  & 60.0  & 3.85797 & 80.0 & 77.22222  & 82.22222  & 16.57017 & 82.8125 & 62.37113 & 93.88489 & 32.66044  \\ \hline
\multicolumn{1}{|c|}{ Partición 3-2 } &  98.94737    & 96.12676  & 66.66667  & 3.78628 & 80.55556 & 66.66667  & 84.44444  & 16.89488 & 82.98969 & 65.10417 & 94.60432 & 38.0697  \\ \hline
\multicolumn{1}{|c|}{ Partición 4-1 } &  98.59155    & 95.78947  & 53.33333  & 4.34525 & 81.66667 & 73.33333  & 80.0  & 17.6753 & 83.85417 & 59.79381 & 94.60432 & 32.38685  \\ \hline
\multicolumn{1}{|c|}{ Partición 4-2 } &  97.89474    & 95.77465  & 73.33333  & 3.71255 & 79.44444 & 72.22222  & 74.44444  & 18.31128 & 82.47423 & 64.58333 & 90.64748 & 39.51833  \\ \hline
\multicolumn{1}{|c|}{ Partición 5-1 } &  98.94366    & 95.4386  & 53.33333  & 4.27863 & 81.11111 & 72.22222  & 75.55556  & 18.13788 & 79.16667 & 63.91753 & 94.60432 & 28.64678  \\ \hline
\multicolumn{1}{|c|}{ Partición 5-2 } &  97.54386    & 96.83099  & 66.66667  & 3.95238 & 78.33333 & 80.0  & 77.77778  & 17.75007 & 83.50515 & 60.9375 & 93.88489 & 31.36983  \\ \hline
\multicolumn{1}{|c|}{ Media } &  98.27798    & 96.20423  & 63.66667  & 3.99974 & 79.88889 & 73.50000  & 80.33333  & 17.45258 & 82.33462 & 63.10835 & 92.51799 & 34.34416  \\ \hline
\end{tabular} }

    \\ \centerline{ILS}
    \resizebox{\textwidth}{!}{ \begin{tabular}{c|c|c|c|c|c|c|c|c|c|c|c|c|}
\cline{2-13}
                                    & \multicolumn{4}{c|}{Wdbc}             & \multicolumn{4}{c|}{Movement\_Libras} & \multicolumn{4}{c|}{Arrhythmia}       \\ \cline{2-13}
                                    & \% clas in & \% clas out & \% red & T & \% clas in & \% clas out & \% red & T & \% clas in & \% clas out & \% red & T \\ \hline
\multicolumn{1}{|c|}{ Partición 1-1 } &  98.23944    & 96.84211  & 33.33333  & 0.7476 & 72.77778 & 70.0  & 32.22222  & 3.77779 & 70.3125 & 65.97938 & 47.48201 & 65.18549  \\ \hline
\multicolumn{1}{|c|}{ Partición 1-2 } &  98.24561    & 96.83099  & 26.66667  & 0.74363 & 73.88889 & 85.55556  & 30.0  & 3.74002 & 72.16495 & 61.45833 & 46.04317 & 53.58842  \\ \hline
\multicolumn{1}{|c|}{ Partición 2-1 } &  98.59155    & 96.49123  & 26.66667  & 0.63564 & 74.44444 & 76.11111  & 33.33333  & 3.31625 & 71.35417 & 61.34021 & 44.2446 & 65.78307  \\ \hline
\multicolumn{1}{|c|}{ Partición 2-2 } &  97.89474    & 96.83099  & 30.0  & 0.89896 & 78.88889 & 72.77778  & 31.11111  & 3.50403 & 72.16495 & 64.0625 & 39.20863 & 51.64644  \\ \hline
\multicolumn{1}{|c|}{ Partición 3-1 } &  98.59155    & 96.14035  & 36.66667  & 0.80554 & 74.44444 & 74.44444  & 34.44444  & 3.43782 & 69.27083 & 63.91753 & 41.00719 & 67.36685  \\ \hline
\multicolumn{1}{|c|}{ Partición 3-2 } &  97.19298    & 95.42254  & 20.0  & 0.78028 & 78.88889 & 73.88889  & 35.55556  & 3.40206 & 70.61856 & 61.45833 & 36.69065 & 46.09727  \\ \hline
\multicolumn{1}{|c|}{ Partición 4-1 } &  97.53521    & 97.54386  & 36.66667  & 0.78197 & 79.44444 & 68.88889  & 32.22222  & 3.18754 & 72.91667 & 64.94845 & 45.68345 & 59.02578  \\ \hline
\multicolumn{1}{|c|}{ Partición 4-2 } &  98.24561    & 96.83099  & 30.0  & 0.70736 & 77.22222 & 72.77778  & 36.66667  & 4.11384 & 69.07216 & 61.97917 & 47.48201 & 45.23494  \\ \hline
\multicolumn{1}{|c|}{ Partición 5-1 } &  98.94366    & 95.4386  & 43.33333  & 1.01458 & 76.11111 & 71.11111  & 38.88889  & 3.96739 & 69.27083 & 64.43299 & 40.64748 & 66.28016  \\ \hline
\multicolumn{1}{|c|}{ Partición 5-2 } &  98.24561    & 95.42254  & 26.66667  & 0.70555 & 73.88889 & 76.66667  & 24.44444  & 3.05619 & 73.71134 & 61.45833 & 43.52518 & 53.40892  \\ \hline
\multicolumn{1}{|c|}{ Media } &  98.17260    & 96.37942  & 31.00000  & 0.78211 & 76.00000 & 74.22222  & 32.88889  & 3.55029 & 71.08570 & 63.10352 & 43.20144 & 57.36173  \\ \hline
\end{tabular} }
 \\
    \newpage
    \centerline{Media}
    \resizebox{\textwidth}{!}{ \begin{tabular}{c|c|c|c|c|c|c|c|c|c|c|c|c|}
\cline{2-13}
                                    & \multicolumn{4}{c|}{Wdbc}             & \multicolumn{4}{c|}{Movement\_Libras} & \multicolumn{4}{c|}{Arrhythmia}       \\ \cline{2-13}
                                    & \% clas in & \% clas out & \% red & T & \% clas in & \% clas out & \% red & T & \% clas in & \% clas out & \% red & T \\ \hline
\multicolumn{1}{|c|}{ KNN } &  96.30974    & 96.48530  & 0.0  & 0.0 & 70.55556 & 70.61111  & 0.0  & 0.0 & 62.85653 & 64.50494 & 0.0 & 0.0  \\ \hline
\multicolumn{1}{|c|}{ SFS } &  96.76563    & 93.60390  & 86.00000  & 0.13475 & 77.38889 & 65.61111  & 90.44444  & 0.71810 & 77.41140 & 68.96209 & 97.44604 & 1.89433  \\ \hline
\multicolumn{1}{|c|}{ BMB } &  97.92637    & 95.67606  & 36.00000  & 1.08608 & 76.66667 & 71.16667  & 48.00000  & 4.38071 & 71.03791 & 65.49077 & 47.76979 & 65.61655  \\ \hline
\multicolumn{1}{|c|}{ GRASP } &  98.27798    & 95.25414  & 63.66667  & 4.05062 & 79.88889 & 70.72222  & 80.33333  & 17.57407 & 82.33462 & 71.49646 & 92.51799 & 34.47605  \\ \hline
\multicolumn{1}{|c|}{ ILS } &  98.17260    & 95.60588  & 31.00000  & 0.80009 & 76.00000 & 70.38889  & 32.88889  & 3.61344 & 71.08570 & 64.09472 & 43.20144 & 57.19147  \\ \hline

\end{tabular} }
\\

    Meter conclusiones.



  \section{Referencias}

  Las referencias utilizadas han sido:
  \begin{itemize}
    \item \emph{Scikit-Learn}: La propia \fnurl{documentación}{http://scikit-learn.org/stable/modules/classes.html} de la biblioteca.
    \item \emph{SciPy}: La propia \fnurl{documentación}{http://docs.scipy.org/doc/scipy/reference/} de la biblioteca.
  \end{itemize}

\end{document}
