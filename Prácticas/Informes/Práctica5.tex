\documentclass[a4paper, 11pt]{article}
\usepackage[utf8]{inputenc}
\usepackage{kvoptions-patch}
\usepackage[titulo={Práctica 5: Algoritmos híbridos}]{estilo}



\begin{document}
  \maketitle
  \tableofcontents
  \newpage

  \section{Descripción del problema}

    El problema a resolver es el problema de \emph{Selección de características}.En el ámbito de la \emph{Ciencia de Datos}, la cantidad de datos a evaluar para obtener buenos resultados es excesivamente grande. Esto nos lleva a la siguiente cuestión: ¿Son todos ellos realmente importantes? ¿Podemos establecer dependencias para eliminar los que no nos aportan información relevante? La respuesta es que sí: en muchas ocasiones, no todos los datos son importantes, o no lo son demasiado. Por ello, se intentará filtrar las características relevantes de un conjunto de datos.

La selección de características tiene varias ventajas: se reduce la complejidad del problema, disminuyendo el tiempo de ejecución. También se aumenta la capacidad de generalización puesto que tenemos menos variables que tener en cuenta, además de conseguir resultados más simples y fáciles de entender e interpretar.

Para conseguir este propósito se deben usar técnicas probabilísticas, ya que es un problema \emph{NP-hard}. Una técnica exhaustiva sería totalmente inviable para cualquier caso de búsqueda medianamente grande. Usaremos metaheurísticas para resolver este problema, aunque también podríamos intentar resolverlo utilizando estadísticos (correlación entre características, medidas de separabilidad o basadas en teoría de información o consistencia, etc).


  \section{Descripción de aplicación de los algoritmos al problema}
    Los elementos comunes de los algoritmos son:
\begin{itemize}
  \item Representación de las soluciones: Se representan las soluciones como vectores 1-dimensionales binarios (los llamaremos \emph{bits} para poder hacer uso de términos como \emph{darle la vuelta a un bit}):

  $$ s = (x_1,x_2,\ldots,x_{n-1},x_n) ; \; x_i \in \{True,False\} \; \forall i \in \{1,2,\ldots,n\} $$
  \item Función objetivo: La función a maximizar es la tasa de clasificación de los datos de entrada:

  $$ tasa\_clas = 100 \cdot \frac{instancias\;bien\;clasificadas}{instancias\;totales} $$

  \item Generación de vecino: La función generadora de vecinos es bien simple. Se toma una solución y se le da la vuelta a uno de sus bits, el cual se escoge aleatoriamente.
  \begin{verbatim}
    Tomar un vector de características "caracteristica"
    indice = generarAleatorio(0, numero_caracteristicas)
    caracteristicas[indice] = not caracteristicas[indice]
  \end{verbatim}
\end{itemize}

  \section{Descripción de la estructura del método de búsqueda}
    Los tres algoritmos se han implementado con el mismo código. Sólo varía un
    booleano en la llamada a la función, que indica si es un algoritmo memético,
    y un real que designa la probabilidad de utilizar la búsqueda local. Si
    es negativo, no es una probabilidad sino una proporción real.
    \begin{itemize}
      \item Algoritmo genético: Igual al genético de la práctica 3 pero con
      la componente híbrida. La condición de hibridación se ha modelado de 3
      formas. La primera es mejorar todos los cromosomas de la población. La
      segunda, mejorar un 10\% aleatorio. La tercera, mejorar los 10\% mejores.
        \begin{verbatim}
          tamaño_cromosoma = número de características
          evaluaciones = 0
          max_evaluaciones = 15000

          probabilidad_cruce = 0.7
          probabilidad_mutacion = 0.001

          tamaño_poblacion = 30

          Si es generacional, numero_seleccionados = tamaño_poblacion
          si no, numero_seleccionados = 2

          numero_cruces = EnteroPorArriba(probabilidad_cruce *
                            numero_seleccionados / 2)

          # Esta probabilidad es la probabilidad de que un cromosoma mute.
          # Esto lo hacemos para evitar trabajar a nivel de gen y generar el
          # mínimo número de aleatorios posible, además de homogeneizar
          # la implementación de ambas variantes del algoritmo
          probabilidad_total_mutacion = probabilidad_mutacion *
                  numero_seleccionadas * tamaño_cromosoma

          # Inicializar la población y evaluarla
          poblacion = Generar "tamaño_poblacion" cromosomas aleatorios
          evaluar(poblacion)
          # Ordenamos la población por tasa en orden creciente
          ordenar(poblacion)

          Mientras evaluaciones < max_evaluaciones
            # Cogemos la solución que está en último lugar,
            # ya que la población está ordenada
            mejor_solucion = poblacion[-1]

            # Selección
            Para cada i en [1,2,...,numero_seleccionados]
              eleccion = Coger 2 individuos aleatorios de la población
              ordenar(poblacion)
              mejor = eleccion[-1]
              seleccionados.añadir(mejor)

            # Cruce
            Para cada i en [1,2,...numero_cruces] tomados de dos en dos
              cruce(seleccionados[i], seleccionados[i+1])

            # Mutación
            Si aleatorio() < probabilidad_mutacion_completa
              cromosoma_mutado = aleatorio()
              gen_mutado = aleatorio()

              flip(cromosoma_mutado, gen_mutado)

            # Actualizar tasas
            evaluar(seleccionados)
            evaluaciones = evaluaciones + numero_seleccionados

            # Componente híbrida
            Para cada i en [1,2,..., numero_seleccionados]
              Si se da la condición de hibridación
                poblacion[i] = BusquedaLocal con solución inicial poblacion[i]

            # Reemplazo
            Para cada i en [1,2,..., numero_seleccionados]
              # Reemplazamos los peores de la población por los mejores seleccionados
              poblacion[i] = seleccionados[-i]
            Si es generacional y mejor_solucion[tasa] > poblacion[0][tasa]
              # Sustituimos la peor solución de la población por la mejor que había
              # Elitismo
              poblacion[0] = mejor_solucion

          ordenar(poblacion)
          mejor_solucion, mejor_tasa = poblacion[-1][cromosoma], poblacion[-1][tasa]
          return mejor_solucion
        \end{verbatim}
        \item Búsqueda Local: Se le ha tenido que añadir una componente
        para almacenar las iteraciones consumidas, ya que al hibridar sólo se
        puede permitir una.
          \begin{verbatim}
            caracteristicas = generaSolAleatoria()
            fin = Falso
            mejor_tasa = coste(caracteristicas)
            evaluaciones = 0

            Mientras haya_mejora y evaluaciones < MAXIMO_EVALUACIONES
            y iteraciones < maximo_iteraciones
              lista_vecinos = aleatoriza(caracteristicas)
              Para cada caracteristica de lista_vecinos no activa

                flip(caracteristicas, indice_caracteristica)
                tasa = coste(caracteristicas)
                flip(caracteristicas, indice_caracteristica)

                si tasa > mejor_tasa:
                  mejor_tasa = tasa
                  flip(caracteristicas,indice_caracteristica)
                  hay_mejora = Verdadero
            evaluaciones_hechas = evaluaciones
              \end{verbatim}
    \end{itemize}


  \section{Descripción del algoritmo de comparación}
    El algoritmo de comparación es un algoritmo greedy: el \emph{Sequential Forward Selection(SFS)}. La idea es muy simple: se parte del conjunto vacío de características (todos los bits a 0) y se recorren todas las características, evaluando la función de coste. La característica que más mejora ofrezca, se coje. Y se vuelve a empezar. Así hasta que ninguna de las características mejore el coste.

\begin{verbatim}
  caracteristicas = (1,2,...,n)
  caracteristicas_seleccionadas = (0,0,...,0,0)
  fin = falso
  mejor_caracteristica = 0


  Mientras mejor_caracteristica != -1
    mejor_tasa = 0
    mejor_caracteristica = -1
    Para cada característica
      tasa = coste(característica)
      Si tasa > mejor_tasa
        mejor_tasa = tasa
        mejor_caracteristica = caracteristica
    Si mejor_caracteristica != -1
      caracteristicas_seleccionadas.añadir(mejor_caracteristica)
\end{verbatim}

  \section{Desarrollo de la práctica}
    En primer lugar, comentar que las bases de datos han sido modificadas en su estructura (que no en sus datos) para que sean homogéneas. Así, se han puesto todas las clases como numéricas (en Wdbc no lo estaban) y se han colocado en la última columna.

    La práctica se ha desarrollado usando el lenguage de programación \emph{Python}, ya que su velocidad de desarrollo es bastante alta. Para intentar lidiar con la lentitud que puede suponer usar un lenguaje interpretado, utilizaremos las librerías \emph{NumPy, SciPy y Scikit-Learn}, que tienen módulos implementados en C (sobre todo \emph{NumPy}) y agilizan bastante los cálculos y el manejo de vectores grandes. Para el KNN con Leave One Out se ha utilizado un módulo que ha desarrollado mi compañero \fnurl{Alejandro García Montoro}{https://github.com/agarciamontoro/metaheuristics}, que usa \emph{CUDA} para agilizar los cálculos usando la GPU.

    Usaremos alguna funcionalidad directa de estas bibliotecas:
    \begin{itemize}
      \item \emph{NumPy}: Generación de números aleatorios y operaciones rápidas sobre vectores.
      \item \emph{SciPy}: Lectura de ficheros ARFF de WEKA.
      \item \emph{Scikit-Learn}: Particionamiento de los datos, se han usado las particiones estratificadas de la validación cruzada 5x2.
      \item \emph{ScorerGPU}: Para el KNN con Leave One Out.
    \end{itemize}

    Esta elección se ha hecho para poder preocuparme sólo y exclusivamente de la implementación de las metaheurísticas.

    Los requisitos para ejecutar mis prácticas son \emph{Python3} (importante que sea la 3), \emph{NumPy}, \emph{SciPy}, \emph{Scikit-Learn} y \emph{CUDA}, por lo que es necesario una gráfica nVidia. En mi plataforma (Archlinux) están disponibles desde su gestor de paquetes.

    Una vez instalados los paquetes, sólo hay que ejecutar la práctica diciéndole al programa los algoritmos que queremos ejecutar. La semilla aleatoria está fijada dentro del código como 12345678 para no inducir a errores. Veamos algunos ejemplos de llamadas a la práctica. Primero notamos que los algoritmos disponibles son:

    \begin{itemize}
      \item -SFS: Ejecuta el algoritmo greedy SFS.
      \item -LS: Ejecuta la Local Search.
      \item -SA: Ejecuta el Simulated Annealing.
      \item -TS: Ejecuta la Tabu Search.
      \item -TSext: Ejecuta la Tabu Search extendida.
      \item -BMB: Ejecuta la Búsqueda Multiarranque Básica.
      \item -GRASP: Ejecuta el GRASP.
      \item -ILS: Ejecuta la Iterated Local Search.
      \item -EGA: Ejecuta el genético estacionario.
      \item -GGA: Ejecuta el genético generacional.
      \item -AM1010: Ejecuta el memético con hibridación total.
      \item -AM1001: Ejecuta el memético con hibridación del 10\% aleatorio.
      \item -AM1001mej: Ejecuta el memético con hibridación del 10\% mejor.
    \end{itemize}

    \begin{verbatim}
      $ python featureSelection.py -TS
    \end{verbatim}
    Se ejecutará la Tabu Search. Pero no sólo se limita el programa a un algoritmo. Si le pasamos varios, los ejecutará en serie uno detrás de otro. Esto ha cambiado desde la práctica anterior por la entrada de \emph{CUDA}, que hay que iniciarlo debidamente y no es tan sencillo de ejecutar cosas en paralelo.

    \begin{verbatim}
      $ python featureSelection.py -EGA -GGA
    \end{verbatim}
    Se ejecutarán EGA y GGA en serie.

    Una vez ejecutado, irán saliendo por pantalla mensajes de este tipo, que proporcionan datos en tiempo real del estado de la ejecución:

    \begin{verbatim}
      INFO:__main__:W - TS - Time elapsed: 2265.526112794876.
      Score: 98.2394337654. Score out: 95.0877192982 Selected features: 15
    \end{verbatim}

    Este mensaje nos dice todo lo necesario: W es la base de datos (Wdbc), TS el algoritmo, el tiempo transcurrido para esta iteración (recordemos que hay 10), el score de entrenamiento, el score de validación y las características seleccionadas.
  \section{Experimentos}
    Como se ha comentado antes, la semilla está fija a 12345678 para no tener problemas de aleatoriedad. El número de evaluaciones máxima de todos los algoritmos es de 15000. Por lo demás, todos los demás parámetros propios de cada algoritmo están tal y como se explica en el guión. \\

    \newpage

    \centerline{KNN}
    \resizebox{\textwidth}{!}{ \begin{tabular}{c|c|c|c|c|c|c|c|c|c|c|c|c|}
\cline{2-13}
                                    & \multicolumn{4}{c|}{Wdbc}             & \multicolumn{4}{c|}{Movement\_Libras} & \multicolumn{4}{c|}{Arrhythmia}       \\ \cline{2-13}
                                    & \% clas in & \% clas out & \% red & T & \% clas in & \% clas out & \% red & T & \% clas in & \% clas out & \% red & T \\ \hline
\multicolumn{1}{|c|}{ Partición 1-1 } &  96.12676    & 96.84211  & 0.0  & 0.0 & 67.77778 & 65.0  & 0.0  & 0.0 & 64.58333 & 66.49485 & 0.0 & 0.0  \\ \hline
\multicolumn{1}{|c|}{ Partición 1-2 } &  96.49123    & 96.83099  & 0.0  & 0.0 & 66.66667 & 80.55556  & 0.0  & 0.0 & 63.40206 & 62.5 & 0.0 & 0.0  \\ \hline
\multicolumn{1}{|c|}{ Partición 2-1 } &  96.12676    & 96.49123  & 0.0  & 0.0 & 66.11111 & 74.44444  & 0.0  & 0.0 & 64.0625 & 64.43299 & 0.0 & 0.0  \\ \hline
\multicolumn{1}{|c|}{ Partición 2-2 } &  95.78947    & 96.12676  & 0.0  & 0.0 & 70.55556 & 68.88889  & 0.0  & 0.0 & 64.94845 & 64.58333 & 0.0 & 0.0  \\ \hline
\multicolumn{1}{|c|}{ Partición 3-1 } &  96.47887    & 95.4386  & 0.0  & 0.0 & 76.11111 & 71.66667  & 0.0  & 0.0 & 64.0625 & 63.91753 & 0.0 & 0.0  \\ \hline
\multicolumn{1}{|c|}{ Partición 3-2 } &  96.8421    & 96.83099  & 0.0  & 0.0 & 66.66667 & 65.55556  & 0.0  & 0.0 & 63.40206 & 64.58333 & 0.0 & 0.0  \\ \hline
\multicolumn{1}{|c|}{ Partición 4-1 } &  97.53521    & 96.14035  & 0.0  & 0.0 & 65.0 & 66.11111  & 0.0  & 0.0 & 67.1875 & 64.43299 & 0.0 & 0.0  \\ \hline
\multicolumn{1}{|c|}{ Partición 4-2 } &  93.33333    & 97.88732  & 0.0  & 0.0 & 70.0 & 72.77778  & 0.0  & 0.0 & 62.8866 & 64.58333 & 0.0 & 0.0  \\ \hline
\multicolumn{1}{|c|}{ Partición 5-1 } &  96.47887    & 96.84211  & 0.0  & 0.0 & 71.66666 & 71.11111  & 0.0  & 0.0 & 65.10417 & 65.97938 & 0.0 & 0.0  \\ \hline
\multicolumn{1}{|c|}{ Partición 5-2 } &  97.89473    & 95.42254  & 0.0  & 0.0 & 65.55555 & 70.0  & 0.0  & 0.0 & 63.40206 & 63.54167 & 0.0 & 0.0  \\ \hline
\multicolumn{1}{|c|}{ Media } &  96.30973    & 96.48530  & 0.00000  & 0.0 & 68.61111 & 70.61111  & 0.00000  & 0.0 & 64.30412 & 64.50494 & 0.00000 & 0.0  \\ \hline
\end{tabular} }
\\
    \centerline{SFS}
    \resizebox{\textwidth}{!}{ \begin{tabular}{c|c|c|c|c|c|c|c|c|c|c|c|c|}
\cline{2-13}
                                    & \multicolumn{4}{c|}{Wdbc}             & \multicolumn{4}{c|}{Movement\_Libras} & \multicolumn{4}{c|}{Arrhythmia}       \\ \cline{2-13}
                                    & \% clas in & \% clas out & \% red & T & \% clas in & \% clas out & \% red & T & \% clas in & \% clas out & \% red & T \\ \hline
\multicolumn{1}{|c|}{ Partición 1-1 } &  97.53521    & 92.2807  & 83.33333  & 23.6847 & 75.0 & 61.11111  & 88.88889  & 86.85704 & 77.08333 & 69.07216 & 98.92086 & 113.45124  \\ \hline
\multicolumn{1}{|c|}{ Partición 1-2 } &  97.54386    & 93.66197  & 86.66667  & 21.27275 & 80.55556 & 73.88889  & 81.11111  & 133.14515 & 75.25773 & 67.70833 & 97.48201 & 221.32613  \\ \hline
\multicolumn{1}{|c|}{ Partición 2-1 } &  95.42254    & 91.22807  & 83.33333  & 24.21075 & 67.22222 & 57.77778  & 92.22222  & 60.01379 & 78.125 & 70.61856 & 97.84173 & 195.53327  \\ \hline
\multicolumn{1}{|c|}{ Partición 2-2 } &  95.78947    & 91.90141  & 93.33333  & 12.81063 & 76.11111 & 73.88889  & 92.22222  & 59.83782 & 79.38144 & 75.0 & 96.04317 & 325.50824  \\ \hline
\multicolumn{1}{|c|}{ Partición 3-1 } &  96.12676    & 92.98246  & 90.0  & 16.16634 & 80.0 & 68.88889  & 87.77778  & 87.70363 & 77.60417 & 70.61856 & 98.92086 & 106.32187  \\ \hline
\multicolumn{1}{|c|}{ Partición 3-2 } &  97.54386    & 96.47887  & 86.66667  & 20.40403 & 70.0 & 63.88889  & 91.11111  & 67.70326 & 80.41237 & 73.95833 & 96.40288 & 292.26432  \\ \hline
\multicolumn{1}{|c|}{ Partición 4-1 } &  98.23943    & 96.49123  & 86.66667  & 19.97793 & 73.33333 & 65.0  & 91.11111  & 68.92823 & 76.5625 & 71.64948 & 98.92086 & 105.00263  \\ \hline
\multicolumn{1}{|c|}{ Partición 4-2 } &  94.73684    & 94.3662  & 90.0  & 16.57442 & 69.44444 & 65.0  & 91.11111  & 73.90462 & 81.4433 & 76.5625 & 97.1223 & 243.91016  \\ \hline
\multicolumn{1}{|c|}{ Partición 5-1 } &  96.47887    & 92.63158  & 90.0  & 16.81309 & 73.33333 & 55.55556  & 94.44444  & 45.33103 & 72.91667 & 66.49485 & 99.28058 & 78.50098  \\ \hline
\multicolumn{1}{|c|}{ Partición 5-2 } &  98.94737    & 93.66197  & 76.66667  & 31.40311 & 62.77778 & 53.33333  & 93.33333  & 57.74021 & 77.83505 & 68.75 & 98.56115 & 134.39024  \\ \hline
\multicolumn{1}{|c|}{ Media } &  96.83642    & 93.56845  & 86.66667  & 20.33178 & 72.77778 & 63.83333  & 90.33333  & 74.11648 & 77.66216 & 71.04328 & 97.94964 & 181.62091  \\ \hline
\end{tabular} }

    \\ \centerline{AM1010}
    \resizebox{\textwidth}{!}{ \begin{tabular}{c|c|c|c|c|c|c|c|c|c|c|c|c|}
\cline{2-13}
                                    & \multicolumn{4}{c|}{Wdbc}             & \multicolumn{4}{c|}{Movement\_Libras} & \multicolumn{4}{c|}{Arrhythmia}       \\ \cline{2-13}
                                    & \% clas in & \% clas out & \% red & T & \% clas in & \% clas out & \% red & T & \% clas in & \% clas out & \% red & T \\ \hline
\multicolumn{1}{|c|}{ Partición 1-1 } &  98.94366    & 96.84211  & 26.66667  & 59.74466 & 72.77778 & 73.33333  & 32.22222  & 154.99307 & 72.91666 & 63.91753 & 33.45324 & 1770.8726  \\ \hline
\multicolumn{1}{|c|}{ Partición 1-2 } &  97.54386    & 96.47887  & 6.66667  & 58.04732 & 76.11111 & 81.11111  & 45.55556  & 169.71776 & 71.13402 & 68.22917 & 28.05755 & 1642.02007  \\ \hline
\multicolumn{1}{|c|}{ Partición 2-1 } &  96.47887    & 97.54386  & 6.66667  & 52.11255 & 80.55556 & 67.77778  & 41.11111  & 167.71163 & 73.4375 & 63.40206 & 37.41007 & 1890.7731  \\ \hline
\multicolumn{1}{|c|}{ Partición 2-2 } &  98.59649    & 95.77465  & 26.66667  & 60.50266 & 76.66666 & 75.55556  & 51.11111  & 169.4549 & 72.68041 & 65.10417 & 29.4964 & 1573.58745  \\ \hline
\multicolumn{1}{|c|}{ Partición 3-1 } &  97.53521    & 96.49123  & 10.0  & 49.39248 & 76.66666 & 69.44444  & 31.11111  & 164.08698 & 73.95834 & 65.46392 & 32.3741 & 1789.77399  \\ \hline
\multicolumn{1}{|c|}{ Partición 3-2 } &  98.24561    & 95.42254  & 20.0  & 56.38092 & 79.44444 & 67.77778  & 40.0  & 163.08225 & 68.5567 & 65.10417 & 35.2518 & 1572.59232  \\ \hline
\multicolumn{1}{|c|}{ Partición 4-1 } &  97.88732    & 96.84211  & 20.0  & 57.13668 & 77.22222 & 70.55556  & 28.88889  & 153.94917 & 70.3125 & 64.43299 & 33.09353 & 1884.23964  \\ \hline
\multicolumn{1}{|c|}{ Partición 4-2 } &  98.24561    & 95.07042  & 23.33333  & 60.78664 & 80.55556 & 65.0  & 30.0  & 152.61948 & 69.58763 & 63.02083 & 32.01439 & 1611.97268  \\ \hline
\multicolumn{1}{|c|}{ Partición 5-1 } &  98.94366    & 94.73684  & 30.0  & 63.54612 & 74.44444 & 72.22222  & 41.11111  & 173.82761 & 71.875 & 67.52577 & 26.97842 & 1790.40634  \\ \hline
\multicolumn{1}{|c|}{ Partición 5-2 } &  97.89474    & 97.88732  & 40.0  & 64.67783 & 77.77778 & 72.77778  & 34.44444  & 153.56865 & 71.64948 & 66.66667 & 37.76978 & 1530.62725  \\ \hline
\multicolumn{1}{|c|}{ Media } &  98.03150    & 96.30900  & 21.00000  & 58.23279 & 77.22222 & 71.55556  & 37.55555  & 162.30115 & 71.61082 & 65.28673 & 32.58993 & 1705.68654  \\ \hline
\end{tabular} }
\\
    \\ \centerline{AM1001}
    \resizebox{\textwidth}{!}{ \begin{tabular}{c|c|c|c|c|c|c|c|c|c|c|c|c|}
\cline{2-13}
                                    & \multicolumn{4}{c|}{Wdbc}             & \multicolumn{4}{c|}{Movement\_Libras} & \multicolumn{4}{c|}{Arrhythmia}       \\ \cline{2-13}
                                    & \% clas in & \% clas out & \% red & T & \% clas in & \% clas out & \% red & T & \% clas in & \% clas out & \% red & T \\ \hline
\multicolumn{1}{|c|}{ Partición 1-1 } &  98.94366    & 96.84211  & 26.66667  & 32.95299 & 73.33334 & 68.33333  & 40.0  & 46.70648 & 72.91666 & 64.94845 & 31.65468 & 321.38808  \\ \hline
\multicolumn{1}{|c|}{ Partición 1-2 } &  98.24561    & 95.07042  & 36.66667  & 31.62474 & 73.88889 & 80.55556  & 53.33333  & 44.39466 & 68.5567 & 67.1875 & 32.73381 & 284.05299  \\ \hline
\multicolumn{1}{|c|}{ Partición 2-1 } &  98.59155    & 94.38596  & 33.33333  & 31.6408 & 76.11111 & 75.0  & 52.22222  & 44.06332 & 71.875 & 67.52577 & 35.97122 & 333.35355  \\ \hline
\multicolumn{1}{|c|}{ Partición 2-2 } &  97.89474    & 95.77465  & 16.66667  & 33.87201 & 81.66666 & 72.77778  & 44.44444  & 44.69038 & 73.71134 & 62.5 & 29.13669 & 274.26005  \\ \hline
\multicolumn{1}{|c|}{ Partición 3-1 } &  97.53521    & 95.4386  & 16.66667  & 34.38936 & 79.44444 & 74.44444  & 36.66667  & 46.74366 & 73.95834 & 64.94845 & 36.69065 & 319.43827  \\ \hline
\multicolumn{1}{|c|}{ Partición 3-2 } &  97.54386    & 96.47887  & 26.66667  & 32.89248 & 77.22222 & 75.55556  & 34.44444  & 46.94561 & 70.1031 & 65.625 & 34.53237 & 269.39632  \\ \hline
\multicolumn{1}{|c|}{ Partición 4-1 } &  98.94366    & 95.08772  & 40.0  & 30.49253 & 78.33334 & 75.55556  & 51.11111  & 46.82215 & 70.3125 & 62.37113 & 35.61151 & 314.2398  \\ \hline
\multicolumn{1}{|c|}{ Partición 4-2 } &  97.54386    & 96.83099  & 30.0  & 32.04749 & 77.77778 & 73.33333  & 38.88889  & 49.37239 & 71.64948 & 67.70833 & 34.53237 & 263.76303  \\ \hline
\multicolumn{1}{|c|}{ Partición 5-1 } &  97.1831    & 96.84211  & 36.66667  & 31.81693 & 79.44444 & 77.22222  & 42.22222  & 46.4231 & 73.4375 & 59.79381 & 34.17266 & 306.47166  \\ \hline
\multicolumn{1}{|c|}{ Partición 5-2 } &  98.59649    & 95.77465  & 13.33333  & 35.09213 & 77.77778 & 71.11111  & 36.66667  & 49.5644 & 71.13402 & 61.97917 & 34.89209 & 260.94701  \\ \hline
\multicolumn{1}{|c|}{ Media } &  98.10217    & 95.85261  & 27.66667  & 32.68215 & 77.50000 & 74.38889  & 43.00000  & 46.57261 & 71.76546 & 64.45876 & 33.99281 & 294.73108  \\ \hline
\end{tabular} }
\\
    \\ \centerline{AM1001mej}
    \resizebox{\textwidth}{!}{ \begin{tabular}{c|c|c|c|c|c|c|c|c|c|c|c|c|}
\cline{2-13}
                                    & \multicolumn{4}{c|}{Wdbc}             & \multicolumn{4}{c|}{Movement\_Libras} & \multicolumn{4}{c|}{Arrhythmia}       \\ \cline{2-13}
                                    & \% clas in & \% clas out & \% red & T & \% clas in & \% clas out & \% red & T & \% clas in & \% clas out & \% red & T \\ \hline
\multicolumn{1}{|c|}{ Partición 1-1 } &  98.59155    & 97.19298  & 30.0  & 32.70323 & 73.88889 & 70.0  & 42.22222  & 48.09855 & 72.39584 & 64.94845 & 33.45324 & 325.10031  \\ \hline
\multicolumn{1}{|c|}{ Partición 1-2 } &  98.59649    & 95.77465  & 26.66667  & 33.15238 & 75.0 & 81.11111  & 56.66667  & 43.79019 & 70.1031 & 64.0625 & 37.41007 & 271.09865  \\ \hline
\multicolumn{1}{|c|}{ Partición 2-1 } &  98.23943    & 96.14035  & 36.66667  & 31.67881 & 82.22222 & 70.55556  & 33.33333  & 49.01478 & 70.83334 & 63.91753 & 30.21583 & 340.51085  \\ \hline
\multicolumn{1}{|c|}{ Partición 2-2 } &  97.89474    & 96.47887  & 33.33333  & 31.97455 & 75.0 & 72.22222  & 42.22222  & 47.7423 & 70.61855 & 66.14583 & 37.76978 & 267.95627  \\ \hline
\multicolumn{1}{|c|}{ Partición 3-1 } &  97.88732    & 94.38596  & 16.66667  & 34.7548 & 75.55556 & 76.11111  & 50.0  & 46.07 & 68.75 & 64.43299 & 30.93525 & 340.91631  \\ \hline
\multicolumn{1}{|c|}{ Partición 3-2 } &  97.19299    & 96.47887  & 26.66667  & 33.23498 & 76.66666 & 73.88889  & 41.11111  & 48.76665 & 74.22681 & 64.0625 & 29.13669 & 306.93667  \\ \hline
\multicolumn{1}{|c|}{ Partición 4-1 } &  96.47887    & 94.03509  & 36.66667  & 32.26469 & 75.0 & 77.22222  & 34.44444  & 50.57397 & 72.91666 & 63.40206 & 30.57554 & 335.57307  \\ \hline
\multicolumn{1}{|c|}{ Partición 4-2 } &  98.59649    & 95.42254  & 23.33333  & 33.60149 & 78.33334 & 71.66667  & 44.44444  & 47.35937 & 70.61855 & 61.97917 & 33.45324 & 283.78787  \\ \hline
\multicolumn{1}{|c|}{ Partición 5-1 } &  97.88732    & 95.08772  & 50.0  & 29.4009 & 76.66666 & 77.77778  & 50.0  & 46.63732 & 72.91666 & 62.37113 & 29.4964 & 325.20679  \\ \hline
\multicolumn{1}{|c|}{ Partición 5-2 } &  99.29825    & 95.42254  & 36.66667  & 32.49137 & 76.11111 & 67.77778  & 53.33333  & 45.1124 & 69.58763 & 65.625 & 26.97842 & 309.52463  \\ \hline
\multicolumn{1}{|c|}{ Media } &  98.06634    & 95.64196  & 31.66667  & 32.52572 & 76.44444 & 73.83333  & 44.77778  & 47.31655 & 71.29671 & 64.09472 & 31.94245 & 310.66114  \\ \hline
\end{tabular} }

    \centerline{Media}
    \resizebox{\textwidth}{!}{ \begin{tabular}{c|c|c|c|c|c|c|c|c|c|c|c|c|}
\cline{2-13}
                                    & \multicolumn{4}{c|}{Wdbc}             & \multicolumn{4}{c|}{Movement\_Libras} & \multicolumn{4}{c|}{Arrhythmia}       \\ \cline{2-13}
                                    & \% clas in & \% clas out & \% red & T & \% clas in & \% clas out & \% red & T & \% clas in & \% clas out & \% red & T \\ \hline
\multicolumn{1}{|c|}{ KNN } &  96.30974    & 96.48530  & 0.0  & 0.0 & 70.55556 & 70.61111  & 0.0  & 0.0 & 62.85653 & 64.50494 & 0.0 & 0.0  \\ \hline
\multicolumn{1}{|c|}{ SFS } &  96.76563    & 93.60390  & 86.00000  & 0.13475 & 77.38889 & 65.61111  & 90.44444  & 0.71810 & 77.41140 & 68.96209 & 97.44604 & 1.89433  \\ \hline
\multicolumn{1}{|c|}{ GRASP } &  98.27798    & 95.25414  & 63.66667  & 4.05062 & 79.88889 & 70.72222  & 80.33333  & 17.57407 & 82.33462 & 71.49646 & 92.51799 & 34.47605  \\ \hline
\multicolumn{1}{|c|}{ EGA } &  97.18829    & 95.43020  & 45.66667  & 25.29545 & 71.83334 & 73.44444  & 53.22222  & 32.77355 & 66.11630 & 64.04156 & 51.33094 & 121.72561  \\ \hline
\multicolumn{1}{|c|}{ UXEGA } &  97.25772    & 94.90277  & 49.33333  & 25.36227 & 73.61111 & 72.61111  & 50.00000  & 32.82589 & 66.37350 & 63.89283 & 51.76259 & 121.40811  \\ \hline
\multicolumn{1}{|c|}{ GGA } &  97.96219    & 95.57030  & 29.66667  & 28.58018 & 76.22222 & 75.38889  & 44.00000  & 33.98448 & 69.53125 & 63.36877 & 34.49640 & 146.04693  \\ \hline
\multicolumn{1}{|c|}{ UXGGA } &  98.03175    & 96.16889  & 27.33333  & 29.21565 & 77.77778 & 72.44444  & 43.88889  & 33.93815 & 73.42193 & 64.71971 & 32.41007 & 151.89248  \\ \hline
\multicolumn{1}{|c|}{ AM1010 } &  98.03150    & 96.30900  & 21.00000  & 58.23279 & 77.22222 & 71.55556  & 37.55555  & 162.30115 & 71.61082 & 65.28673 & 32.58993 & 1705.68654  \\ \hline
\multicolumn{1}{|c|}{ AM1001 } &  98.10217    & 95.85261  & 27.66667  & 32.68215 & 77.50000 & 74.38889  & 43.00000  & 46.57261 & 71.76546 & 64.45876 & 33.99281 & 294.73108  \\ \hline


\end{tabular} }
\\

    En primer lugar, comentamos los 2 algoritmos principales: el KNN y el SFS. El primero de ellos obtiene una buena tasa de acierto en la base de datos pequeña, \emph{Wdbc}, y cuanto más grande es la base de datos, más solapamiento se produce y menos tasa de acierto va teniendo.

    El SFS se caracteriza por tener una tasa de acierto que sólo mejora en la base de datos más grande a la del KNN, pero tiene una tasa de reducción bastante alta, ya que al partir de una solución sin características y al parar en cuanto no hay mejora, es muy fácil que se quede con muy pocas características.

    Ahora evaluamos las metaheurísticas implementadas. Los meméticos son capaces de superar al GRASP en términos de clasificación en las bases de datos pequeñas. En \emph{Arrhythmia} sigue ganando el GRASP con un par de puntos de ventaja.

    En particular, los meméticos no obtienen una mejora sustancial en clasficación con respecto a los genéticos sin hibridar. Sólo en la base de datos más grande, \emph{Arrhythmia}, consiguen mejorar el resultado en clasificación de sus predecesores. Esto podría ser porque la diversidad que teneran los genéticos por sí mismos es suficiente para los problemas más pequeños, pero no son capaces de explotar bien según qué zonas del espacio de búsqueda si éste es muy grande. Por esta razón, es posible que este tipo de algoritmos funcionara mejor en bases de datos más grandes todavía.

    El tiempo extra que tardan estos algoritmos no compensa demasiado la ganancia que tienen, al menos en las bases de datos pequeñas. En \emph{Arrhythmia} sí que se debe intentar rascar cualquier punto de más en clasificación puesto que de por sí la tasa de clasificación no es muy buena.

    El número de individuos a mejorar con hibridación altera el equilibrio diversidad-explotación que tanto se trabaja en este campo, pero no se consiguen resultados muy dispares, ya que el equilibrio tiende más a la explotación al utilizar la búsqueda local.

    En esta práctica yo me quedaría personalmente con el genético estacionario con cruce uniforme. El cruce uniforme está muy usado en la literatura por ser bastante eficiente en su cometido, ya que introduce más diversidad que el cruce en dos puntos. El estacionario hemos visto que tiene muy poca pérdida con respecto al generacional, con la ventaja de que tiene tasa de reducción de columnas notable. Un poco menos de precisión, pero el tamaño del problema se hará notablemente más pequeño.

  \newpage
  \section{Referencias}

  Las referencias utilizadas han sido:
  \begin{itemize}
    \item \emph{Scikit-Learn}: La propia \fnurl{documentación}{http://scikit-learn.org/stable/modules/classes.html} de la biblioteca.
    \item \emph{SciPy}: La propia \fnurl{documentación}{http://docs.scipy.org/doc/scipy/reference/} de la biblioteca.
  \end{itemize}

\end{document}
